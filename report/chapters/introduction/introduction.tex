\startfirstchapter{Introduction}

% An introduction about GitHub
GitHub\footnote{\url{https://www.github.com}} is a popular source code hosting site. It enhanced the version control tool Git\footnote{\url{https://en.wikipedia.org/wiki/Git_(software)}} by combining the features of both software engineering and social networking, including event feeds, pull requests, code reviews,
and an issue tracking mechanism\footnote{\url{https://github.com/features}}, which has greatly facilitate developers' collaboration, communication and coordination.  

% Introduce Gists
Gist is one of the many features of GitHub, which allows users to instantly share code, notes, and snippets. GitHub illustrates\footnote{\url{https://help.github.com/articles/about-gists/}} Gist as follows:

\textsl{``Gists are a great way to share your work. You can share single files, parts of files, or full applications...Every gist is a Git repository, which means that it can be forked, cloned, and manipulated in every way.''}

% motivations
Since how a technology is supposed to be used might differ from how it is actually used, it would be interesting to know how Gist is actually being used, and if it's exactly what GitHub expects or the users are finding innovative ways. However, Gist has been paid little attention by researchers, and there's no existing Gists dataset available for research. Thus, a lot of questions about Gists remain unanswered. For example, how many GitHub users are using gists? Do users collaborate on Gists? What are the contents of Gists about? How large is a Gist? 

A noteworthy trend is that lots of third-party applications have been developed to support creation, management and sharing of Gists. For example, both \textit{GistBox}\footnote{\url{http://www.gistboxapp.com/}} and \textit{Gisto}\footnote{\url{http://www.gistoapp.com/}} are applications that help users better organize their Gists by adding additional features such as searching, tagging and sharing Gists; \textit{GistBox Clipper} allows users to create Gists from any web page; many blog sites like \textit{WordPress}\footnote{\url{http://www.wordpress.com/}} support Gists embedding either by its URL or its ID\footnote{\url{http://en.support.wordpress.com/gist/}}; many popular IDEs or text editors support Gists creation through plugins, such as Visual Studio\footnote{\url{https://marketplace.visualstudio.com/items/dbankier.vscode-gist}}, \textit{IntelliJ IDEA}\footnote{\url{https://www.jetbrains.com/idea/help/creating-gists.html}}, \textit{sublime text}\footnote{\url{https://github.com/bgreenlee/sublime-github}}, etc. All these threads imply that gists are gaining popularity among GitHub users. It's worth to explore Gists and how developers are using them.

% research questions
In this report, an empirical study of GitHub Gists is presented. With the goal of getting a picture of what Gists look like and how they are being used, the following two research questions were addressed:

\begin{itemize}

  \item {\bf RQ1. What do Gists look like?}
  
  Although gists are gaining popularity among GitHub users, the picture of what Gists are like is still not clear. 

  \item {\bf RQ2. How are Gists being used?}
  
  GitHub Gists was initially developed for users to write and share code snippets, but Gists could be used in unexpected ways. Thus it's worthy to find out various ways of using Gists which could provide good recommendations and suggestions on how GitHub Gists could be better used.
  
\end{itemize}
To answer {\bf RQ1}, both qualitative and quantitative analyses were performed around Gists contents and metadata. For the quantitative part, 618,393 Gists and 793,891 files contained in those Gists were downloaded in order to understand what Gists look like. Also, randomly selected 562,993 GitHub users were analyzed to know the proportion of GitHub users that are using Gists. For the qualitative part of the study, a manual inspection was performed on 400 randomly selected Gists. To answer {\bf RQ2}, the Web was searched for evidence of how users described their use of Gists, which included a search of Websites (including blog posts) and Twitter feeds.

Our results can be summarized as follows. Only a small portion of GitHub users use Gists. Gists are usually small snippets of source code, and they usually contain only one file. Gists content has a wide variety including data files (such as JSON, XML), binary files (such as images, audios, videos), and text files of various kinds (such as blogs, logs, letters, restaurant menus). Majority of Gists are not updated frequently, and users rarely fork or comment other users' Gists. People are actually using Gists for various purposes, from creating reusable components, sharing files, saving notes or tops, to drafting writings.

The main contribution of this study is that it provided initial insights into what Gists are and how they are used through exploratory data analysis. Though it's just scratching the surface of Gists, the study result can definitely inspire other researchers to conduct further research on Gists.