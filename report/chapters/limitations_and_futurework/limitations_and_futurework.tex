\startchapter{Limitations and Future Work}

\section{Limitations}
I have to admit there do exist threats to the validity of our results. 

First of all, though private Gists are available to whoever has the Gist URL, they cannot be queried through the APIs GitHub provides. Therefore, we can only perform the analysis against public Gists, which introduces limitations to our results. The answer to our research questions for private Gists might differ from public ones.

Besides, the manual labeling in our qualitative analysis may introduce errors into our results. One researcher conducted the labeling by manually reading and interpreting the Gist content. The misinterpretation can introduce errors. 

Also, since we have no control of how the search engines of Google and Twitter work, our searched results really depend on when search was performed. This dependence makes the results transient and likely to change. 

\section{Future Work}

This study is exploratory. I was just scratching the surface of GitHub Gists. One aspect I didn't reach is what users think about Gists. In the future, interviews of users should be involved. Some interview questions could be: what motivates a user to create a Gist? When do users choose to use Gists instead of GitHub repositories? What determines if a user create a public Gist or a private one? Do they collaborate much on Gists?

Lots of users use Gists to save reusable components, but where do these reusable components from? Are they from other people's blogs? Or the answers of Q\&A websites like StackOverflow? Or GitHub repositories? 

There also exist other tools for users to save and share code snippets. A comparison between Gists and these tools could also be an interesting research topic.