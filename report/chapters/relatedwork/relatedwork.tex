
\startchapter{Related Works}
\label{chapter:relatedworks}

\newlength{\savedunitlength}
\setlength{\unitlength}{2em}

\section{Version Control History and GitHub}

%CVCS
For the large and complex software projects, it is a challenge to well coordinate the collaboration between all the developers and to manage each developer's contribution. Based on this demand, Version Control System (VCS) was developed to record changes to a file or set of files over time so that a specific version can be recalled later. \cite{chacon2009pro} The traditional Centralized Version Control System (CVCS) requires developers to commit a change to a central repository, merge and resolve the conflicts. This mechanism leads to the downside where the entire history of the project lies in the central database, and if the central database becomes corrupted, all the history data would be lost. \cite{chacon2009pro}\cite{5071408} 

%DVCS
Distributed Version Control System (DVCS) relaxes the requirement of CVCSs to have a central, master repository. In a DVCS, each developer owns a first-class repository of its own right and contains the entire history data,\cite{5071408} thus making developing, releasing and coordinating large open source software projects much more flexible than traditional CVCS. \cite{6607694}

%Git
Among all these DVCSs, Git has gained the most momentum. It began in 2005 as a revision management system used for coordinating the Linux kernel's development. Over the years, Git has evolved by ``leaps and bounds'' due to its functionality, portability, efficiency, and rich third-party adoption.\cite{6188603}

%Applications based on Git
Using Git as a back end to host open source projects, many Web-based applications enhanced project management functionality by adding rich user-friendly user interfaces, which provides a convenient way for developers to set up repositories, clone existing projects and commit their contributions.\cite{6188603} These applications also lay more emphasis on the social aspect of software engineering. 

%GitHub
Among all these applications, GitHub is the most popular one with more than 12.2 million users\footnote{\url{https://github.com/about/press}}. GitHub not only allows users to \textit{star}\footnote{\url{https://help.github.com/articles/about-stars/}} or \textit{watch}\footnote{\url{https://help.github.com/articles/watching-repositories/}} repositories to keep track of projects they find interesting, but also allows users to follow other users to see what other people are working on and who they are connecting with. GitHub also supports team management by \textit{organizations}, projects discovery by \textit{explore}, bugs tracing by \textit{issue}, etc.\footnote{\url{https://help.github.com/articles/be-social}} By integrating these social features into version control system, the communication and coordination among developers gets greatly enhanced.\cite{6357175}

\section{Recent Research on GitHub}
GitHub introduces a new open source environment that has given rise to research from many different angles. GitHub is a huge data pool that contains not only numerous projects but also developers' profiles and developers' activities such as their contributions to projects and their interactions with other developers. More and more researchers have jumped into this data pool, trying to discover some interesting patterns or good stories in terms of either software engineering or social networking. 

Some researchers attempt to help employers find technical experts by looking into developers' profile and their activities.\cite{6336698}\cite{Marlow:2013:ATS:2441776.2441794}\cite{Venkataramani:2013:DTE:2487788.2487832}\cite{6671295} Others have focused on the source code in project repositories. For example, Bissyande et al. (2013) took advantage of rich data on GitHub by examining the ``popularity'', ``interoperability'', and ``impact'' of various programming languages measured in different ways, such as lines of code, development teams, issues, etc.\cite{6649842} There are also some researchers who tried to discover patterns in developers' collaboration and interaction, such as how developers assess each other and find proper collaborators\cite{Majumder:2012:CTF:2339530.2339690}\cite{Marlow:2013:IFO:2441776.2441792}\cite{Singer:2013:MAS:2441776.2441791}, the herding phenomena on GitHub\cite{Choi:2013:HOS:2441955.2441989}, the relation between developers' behavior on GitHub and other Q\&A websites like StackOverflow\footnote{\url{http://www.stackoverflow.com/}}\cite{6693332}, etc.

However, no scientific studies focusing on GitHub Gists have been found, which makes it a brand new research spot.

\section{GitHub Gists}

\subsection{Related Tools}
There are some other snippets management tools similar to GitHub Gists such as \textit{pastebin}\footnote{\url{http://pastebin.com/}}, \textit{snipt}\footnote{\url{https://snipt.net/}}, \textit{codepen}\footnote{\url{http://codepen.io/}}, \textit{dabblet}\footnote{\url{http://dabblet.com/}}, etc. However, GitHub Gists is the only one that manages snippets using version control. Over these years, GitHub Gists keeps being updated to be more user friendly as is stated in their blogs\footnote{\url{https://github.com/blog/1837-change-the-visibility-of-your-gists}} \footnote{\url{https://github.com/blog/1850-gist-design-update}}.

\subsection{The Features of Gists}

Gists are small snippets of code or text that utilizes the version control tool Git for creation or management. Wikipedia provides a good explanation of the benefits of Gists as follows\footnote{\url{http://en.wikipedia.org/wiki/GitHub\#Gist}}: 

\textsl{``Gist builds upon that idea by adding version control for code snippets, easy forking, and SSL encryption for private pastes. Because each `gist' is its own Git repository, multiple code snippets can be contained in a single paste and they and be pushed and pulled using Git. Further, forked code can be pushed back to the original author in the form of a patch, so pastes can become more like mini-projects.''}

In addition, GitHub also provides a powerful Web-based editor to create or modify Gists, which makes it possible to work on Gists without Git. It also supports comments on Gists and provides a Web service that makes Gists embeddable in a Web page. All these features make Gists very flexible, functional and user friendly, helping users better manage their gists and share them.