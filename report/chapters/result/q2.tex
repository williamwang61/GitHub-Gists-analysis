\section{RQ2. How are Gists being used?}

We answered this question based on the searched result of two sources: Websites and Twitter. We used ``Github Gist'' as keywords to search for related Web pages or tweets. While reading these content, we took notes in terms of suggested Gist uses. Finally we combined the results from Web pages and tweets, and classified them into several categories. Where appropriate, some quotations from the sampled Websites and Tweets are provided to illustrate our findings which are as follows. 

\subsubsection{Being embedded in blogs}

One of the biggest blog websites, WordPress\footnote{\url{https://wordpress.com/}} has supported embedding Gists in the blogs\footnote{\url{http://crunchify.com/how-to-embed-and-share-github-gists-on-your-wordpress-blog}}. Apart from that, Many other blogging platforms like Medium\footnote{\url{https://medium.com/}} also supporting such embedding soon.\footnote{\url{https://medium.com/the-story/yes-we-get-the-gist-1c2a27cdfc22}} This makes it easy to manage the code in blogs by saving it in Gists and simply referencing it in blogs by ID or URL.

\textsl{``I think I might start using GitHub Gists in my blog posts. Thinking about how to integrate it smoothly.''}\footnote{\url{https://twitter.com/BenNadel/status/157495925231714304}}

\subsubsection{Version-controlled list}

Gists can serve as version-controlled lists by taking advantage of the markdown rendering. Also the version control system can help to keep track of the change history of the lists, adding functionality at no cost. One good example is a popular blog that has been widely tweeted on Twitter. The blog illustrates how to maintain a to-do list using a private GitHub Gist authored by Carl Sednaoui\footnote{\url{http://carlsednaoui.com/post/70299468325/the-best-to-do-list-a-private-gist}}.

\textsl{``GitHub Gists are a great way to keep version-controlled lists (in this case, US states I've visited) \url{https://gist.github.com/dliggat/11003570}''}\footnote{\url{https://twitter.com/dliggat/status/458090816930848768}}

\textsl{``I’m a HUGE fan of todo lists. They help me stay organized, prioritize my day and add structure to an otherwise chaotic day. I recently discovered what appears to be the best yet simplest way to keep a to-do list: a GitHub Gist.''}\footnote{\url{http://www.carlsednaoui.com/post/70299468325/the-best-to-do-list-a-private-gist}}

\subsubsection{Saving notes/tips}

Some people use Gists to save notes or tips such as technical information, learning outcomes tips, etc. 

\textsl{``I've been saving my learning as gists because blogging is a barrier: \url{https://gist.github.com/Greg-Boggs}''}\footnote{\url{https://twitter.com/gregory_boggs/status/483111455550877697}}

\textsl{``Useful fiddles and gists collected from \#AngularJS forum discussions \url{https://github.com/angular/angular.js/wiki/JSFiddle-Examples...}''}\footnote{\url{https://twitter.com/AppsHybrid/status/481901294563901440}}

\subsubsection{Sharing files}

Gists can also be used to share files with other people. Even if a Gist is private, it can still be visited by whoever has its URL. Sometimes private Gists can be a perfect way to share files. The following conversation on Twitter is a good example of this use.

\noindent User A to user B: \textsl{``Can you show me what your application.xml looks like for your ios+android game? Need to know what info is needed to compile.''}\\
User B replied user A: \textsl{``I can put it on GitHub Gists sometime soon, remind me in a few days if I forget...''}\footnote{\url{https://twitter.com/stvr_tweets/status/483092598639185920}}

\subsubsection{Drafting}

Gists are also a perfect place to write a draft. For example, a user uses a Gist\footnote{\url{https://gist.github.com/danbri/58db1297f3b488da9f86}} to draft the wiki for a Google project \textit{Cayley}\footnote{\url{https://github.com/google/cayley/wiki}}. This use can also be suggested by the following tweet: 

\textsl{``Every once in a while I think I wish I could ``draft'' a pull-request, issue, or comment on GitHub, then I remember that private Gists exist.''} \footnote{\url{https://twitter.com/nuclearsandwich/status/249213040610910209}}
