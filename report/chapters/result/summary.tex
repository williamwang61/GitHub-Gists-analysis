\section{Summary}

Based on all the analysis above, our observations can be summarized as the following themes.

\subsubsection{Majority of GitHub users don't use public Gists.}

For those randomly sampled GitHub users, only 5.82\% of them have at least one public Gist. For those who have public Gists, half of them have less than 5 Gists. Also, 86.79\% of those sampled Gists only contain 1 file. All these results show that GitHub users generally don't use public Gists much. 

\subsubsection{Gists usually contain small files.}

According to the metadata provided by GitHub, I found that majority of Gist files were around 1KB. The median size was 714 Bytes. As for all the text files, 87.1\% of them have less than 100 lines, and the median number of lines is 23. I also calculated the SLOC for all the source files, and 92.1\% of them have less than 100 SLOCs with median as 18 SLOCs. 

\subsubsection{Gists content shows a great variety.}

There are 113 MIME types and 257 programming languages discovered from all the Gist files downloaded. Although majority of Gist files are source files, I also observed data files (such as JSON, CSV, XML) and binary files (such as images, audios and even videos) in Gists. During our manual inspection against Gists content, I also found blogs, letters, and even restaurant menus. 

\subsubsection{Users don't collaborate on Gists.}
Gists are found mostly personal artifacts. Among all of the sampled Gists, 62.9\% of them had a single commit (by the owner obviously), and only 1.85\% had more than 10 commits. The creation and the latest update of 83.7\% of Gists happened at the same day. 94.6\% of Gists have never been forked by other users, and only 7.3\% of Gists have been commented once or more. 

\subsubsection{Gists are being used for a variety of purposes.}
Based on our searched results from Websites and Twitter, a great variety was discovered in terms of what people use Gists for: some people use it to create components that can be embedded in their blogs; some use it to keep track of some activities by a version controlled list; some save their notes such as learning outcomes in Gists; some share content with others simply by Gist URLs; some use it to draft their writings.