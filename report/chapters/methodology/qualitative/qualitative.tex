\section{Qualitative Analysis of Gists Content}

In order to make qualitative analysis of actual contents of Gist files, we randomly chose 400 from those 618,393 Gists and tried to manually analyze them by coding them. For each of these 400 Gists, we started reading through it to get a general idea of its content and then labeled it. Since a Gist is composed of one or more files, we used two strategies for labeling a Gist: analyze the content of the Gist as a whole, and determine the relationships between all the files of the Gist.  

\begin{itemize}

  \item Strategy 1: labeling based on content of the whole Gist

  We manually assigned labels to each Gist based on its description, key words in Gist files and personal experience. A Gist could have more than one labels and these labels are not necessarily exclusive as are shown in Table \ref{tb:gistcontentlabels}.

\begin{table}[!htb]
 \begin{center}
 \begin{tabular}{ll} 
   \textbf{Labels of Gists Content}	&	\textbf{Description}\\ \hline
   Code & Source code.\\
   Test & Test code.\\ 
   Class & Only a class is defined.\\ 
   Template & Coding example/pattern.\\ 
   Command & Commands used in shell.\\ 
   Function & Only one or several functions are defined.\\ 
   Fragment & Several lines of code without complete functions/classes.\\
   Note & Script without source code.\\ 
   Log & Log files.\\ 
   Configuration & Configure files used for executing code.\\ 
   Diff & Diff files used for visualizing the changes in a file.\\ 
   Documentation & Text tutorial documentations.\\ 
   Data & Data stored in json, csv or other forms.\\ 
   Blog & Technical blog in narrative format.\\ 
   Non-technical & Notes without any technical content.\\ \hline
 \end{tabular}
 \end{center}
 \caption{Description of Gist labels in terms of content.}
 \label{tb:gistcontentlabels}
\end{table}

  \item Strategy 2: labeling based on files relationships in a Gist

  Since a Gist may contain more than one file, we assumed there could exist some relationships between the files in each Gist. These relationships are listed in Table \ref{tb:gistfilerelationshiplabels}.

\begin{table}[!htb]
 \begin{center}
 \begin{tabular}{ll} 
    \textbf{Labels of Gist}	&	\\ 
    \textbf{Files Relationships}	& \textbf{Description}	\\  \hline
    Single File & The Gist contains one file.\\
    Independent & Files in a Gist are independent.\\
    Reference & One file refers to another file.\\ 
    Generation & One file is the input/output of another file.\\ 
    Test & One file is to test another file.\\ 
    Attachment & One file is the configuration or information of another file.\\ \hline
 \end{tabular}
 \end{center}
 \caption{Description of Gist labels in terms of files relationships.}
 \label{tb:gistfilerelationshiplabels}
\end{table}

\end{itemize}

After finishing the data coding, we made statistical analysis of these labels.

\section{Qualitative Analysis of Users' Discussion Regarding Gists}

We also searched the Web pages and tweets on Twitter for users' description about how Gists were used. We considered all relevant Web pages, either official or unofficial, as well as the most recent comments at the time this study was conducted. 

\begin{enumerate}
	\item Web Pages. In order to find relevant Web pages, we used Google Search Engine to perform the searching using the keyword ``GitHub Gist''. For each of the search results, we manually read through the Web page, extracted the information about how Gists were used, and took notes.

	\item Twitter Postings. As is known, Twitter has become a big information gathering pool, providing much data for researchers. Recently, Twitter has come to be a more and more popular platform for developers to learn, share, discuss technical questions, so we chose Twitter as a information source to help understand how people are using gists. We performed the query using ``GitHub Gist'' as the searching keywords on Twitter Web search interface, and read all the searched tweets one by one. While reading these data, we took notes in terms of how they used Gists suggested in the posting. 
\end{enumerate}