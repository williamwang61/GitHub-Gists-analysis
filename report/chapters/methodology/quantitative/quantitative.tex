\section{Quantitative Analysis of Gists Metadata and Content}

With the goal of finding some patterns and characteristics of Gists so as to get a overview of what Gists look like. I quantitatively analyzed the Gists data that were downloaded. Not only did I analyze the metadata of the sampled 618,393 Gists but also inspected the actual content of the 793,891 Gist files.

\subsection{Analysis of Gists Metadata}

The quantitative analysis for Gists metadata was based on the a series of sub-questions:
\begin{itemize}
	\item How many Gists does each user own?

	\item How many files are there in a Gist?

	\item How large is a Gist?

	\item What languages are used in Gists?

	\item How many commits/forks/comments are there per Gist?

	\item How often do users update their Gists?
\end{itemize}

\subsection{Analysis of Gists Files}

I first analyzed the size metadata of all Gists files downloaded. Then for each file whose MIME type is ``text'' or ``application'',  I calculated its number of lines. Further analysis was made by taking advantage of the source code analyzing tool \textit{CLOC}\footnote{\url{https://github.com/AlDanial/cloc}}. \textit{CLOC} is able to detect uniqueness of text files based on \textit{Digest::MD5}\footnote{\url{http://search.cpan.org/dist/Digest-MD5/MD5.pm}}, as well as calculate source lines of code (SLOC\footnote{\url{https://en.wikipedia.org/wiki/Source_lines_of_code}}) for source files in many widely used programming languages\footnote{\url{https://github.com/AlDanial/cloc\#Languages}}. 