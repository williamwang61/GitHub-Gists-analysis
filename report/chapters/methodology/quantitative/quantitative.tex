\section{Quantitative Analysis of Gists Metadata and Content}

Based on the obtained gists metadata and contents, we made a series of quantitative analysis using statistics, trying to find some patterns and characteristics of gists so as to get a overview of what gists look like. Not only did we analyze the metadata of the sampled 618,393 Gists and Gist files but also inspected the actual content of the 793,891 Gist files.

\subsection{Analysis of Gists Metadata}

We made statistical analysis for Gists metadata based on the a series of sub-questions:
\begin{itemize}
	\item How many Gists does each user own?

	\item How many files are there in a Gist?

	\item How large is a Gist usually?

	\item What languages are used in Gists?

	\item How many commits/forks/comments are there per Gist?

	\item How often do users update their Gists?
\end{itemize}

\subsection{Analysis of Gists Files}

As for Gist files, apart from looking into their size metadata, we also calculated the number of lines for each file of text or application MIME type. Then further analysis was made by taking advantage of the source code analyzing tool \textit{CLOC}\footnote{\url{https://github.com/AlDanial/cloc}}. \textit{CLOC} is able to detect uniqueness of text files, as well as calculate source lines of code (SLOC\footnote{\url{https://en.wikipedia.org/wiki/Source_lines_of_code}}) for source files in many widely used programming languages\footnote{\url{https://github.com/AlDanial/cloc\#Languages}}. 